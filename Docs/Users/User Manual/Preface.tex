\sct is a C\# library for scientific computing applications.
It contains a set of frameworks for a number of areas of scientific
computing, such as optimization, simulation, graph searching, 
probabilities and statistics, language processing, numerical algorithms,
computational geometry and number theory. It also provides tools for
testing and comparison of algorithms and for the generation of 
reports in several formats.

The main goal of \sct is to provide a set of tools for
the scientific community to develop algorithms in .NET for 
a very broad range of applications. Despite
being written entirely in C\# it aims to provide efficient
implementations, based on parallel, distributed and GPU programming
techniques. 

This report describes the main functionalities built into
the library, and provides a guide for its use and extension.
It is by no mean a complete nor updated guide. \sct
is always changing, and the best place to find up to date and
extensive information on the nasty details of the library
is on its \href{http://scientifictools.codeplex.com}{website}.

The rest of the report is divided into chapters, one for each
of the \sct main \emph{namespaces}. Each chapter builds upon
one or more examples in a tutorial-like fashion. There are
of course many details that escape these examples. For some
of the most important details, footnotes or other indications will
be given.

%========================================================================
\section*{Prerequisites}
\addcontentsline{toc}{section}{Prerequisites}
%========================================================================

\sct is a library written in C\# for scientific computing. As
such, this manual considers on behalf of the reader a working
understanding both of the C\# language and the scientific topics
covered by the library. 

The algorithms and data structures implemented in the library
are in most cases well known in the scientific community.
In such cases, the first time a given algorithm, data structure
or any other important concept is mentioned in this manual, a
reference will be provided to the corresponding 
\href{http://en.wikipedia.org}{Wikipedia} page. All non-standard
modifications will be explained in detail proportional to
how strange or new is the given modification, and when possible
references will be provided from the corresponding papers or
websites.

%========================================================================
\section*{About the Source Code}
\addcontentsline{toc}{section}{About the Source Code}
%========================================================================

All code used throughout the manual is written in C\#, adhering
to the style used in by the developers of the library. Being a 
.NET library, all functionality can be used in any CLR compliant
language. Most the examples used in the manual are shipped
with the source code for the library. 
Source code will be formatted in \verb|monospace| font, with
the corresponding syntax highlight.

The \sct library has been designed following modern
design patterns and coding styles. Throughout the library we have
tried to be consistent with a unified style, which will be used in
the code examples for this manual. Being a scientific library, we
have tried also to provide support for some flavoured syntax styles
widely used in some scientific packages, such as Matlab and Octave.
We have also tried to supply a syntax as close as possible to that
used by academicians in the corresponding areas.

We have implemented this syntax sugar with heavy use of operators
overloading and other fluid interfaces techniques. This means 
that there are almost always two ways of doing things, with
the usual C\# syntax (dot notation and such) and with the flavoured
syntax which may involve using some strange operators and 
method names, in order to get a code that looks familiar
to the mathematician eye. 

We have designed the library this way because we feel it is
more natural for the scientist, but it could be weird for the
programmer used to the clean dot notation syntax. If you ever
feel uncomfortable using the flavoured syntax, you can
always find a way of doing the same thing using standard
C\# syntax. 

We particularly prefer the syntax sugar
\footnote{There is a common joke that says: \emph{Syntax sugar
causes cancer of the semicolon}.} whenever possible, because
we designed the library this way, it has become a
common language for the library developers, and in the code examples
you'll find it anywhere. In any case, in this manual we will
always provide hints of how to circumvent the syntax sugar
in favour of a more C\#-ist syntax.

%========================================================================
\section*{What this Manual Is Not}
\addcontentsline{toc}{section}{What this Manual Is Not}
%========================================================================

This manual is \textbf{not} a reference for algorithms and
data structures. You will not find implementation details for the
algorithms, nor proves of asymptotic complexity, or any other
theoretical details of the implemented tools, other than the very
basic definitions.

This manual does \textbf{not} explain which algorithm or technique
is better in which case, nor provides comparisons between algorithms 
other than (perhaps) some information about our particular implementation
that we think you may need to know. For instance, we will never say
that \emph{QuickSort} is faster than \emph{MergeSort} (well, I just did, but
that is not the point). We could say, however, that \textbf{our} \emph{QuickSort}
implementation is faster than \textbf{our} \emph{MergeSort} implementation.

You \textbf{will} find, however, links to
relevant places where you can start doing some research
about the techniques implemented. You can also look at the
source code in any moment, where you'll find all implementation
details you want (is all in there, isn't it?). Also, the
examples have been designed trying to choose interesting or at least
classic problems that can be solved using the corresponding
techniques. 

But always keep in mind that we are just giving you
a hammer, it is your choice how to use it: to build a house,
to bake some cookies, or to break someone's head. The examples
we provide will almost always be about house building, in fewer
cases about cooking, but \emph{almost} never about head smashing. 