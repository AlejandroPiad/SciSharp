The root \emph{namespace} in \sct contains
basic tools used by the rest of the framework. This
includes the basic matrix and vector operations,
function definition and composition, 
definition of exceptions specific to the
framework, some exension methods for \texttt{IEnumerable<T>},
logging and debugging tools, and some basic data types.
The \texttt{Vector} and \texttt{Matrix} classes provide
algebraic operations on these data types. The \texttt{Function}
static class and related classes and interfaces provide
the functionality for function definition, composition and
differentiation.

Other important classes in the root \emph{namespace} are
\texttt{Engine}, \texttt{Debug}, \texttt{Logger} and
\texttt{Serializers}.
These are static classes which store configuration
parameters for the rest of the framework. The \texttt{Engine}
class deals with numerical comparisons and the configurations
for the parallel, distributed and GPU implementations of the
algorithms which support these features. The \texttt{Debug}
class provides methods to track the execution of algorithms.
The \texttt{Logger} class provides methods to generate
logging data. The \texttt{Serializers} class provides methods
for saving and loading data in various formats.

%========================================================================
\section{Algebraic Operations}
%========================================================================

The \typeref{Vector}{ScientificTools.Vector} and 
\typeref{Matrix}{ScientificTools.Matrix} classes provide an
object-oriented abstraction of the corresponding
mathematical entities. Both classes provide the corresponding
definitions for arithmetic operations (using static methods, instance
methods and operator overloading) and indexers.

%========================================================================
\section{The \texttt{Engine} Class}
%========================================================================

%========================================================================
\section{Debugging and Logging}
%========================================================================

%========================================================================
\section{Serialization}
%========================================================================

%========================================================================
\section{Configuration Contexts}
%========================================================================

The four classes seen in the previous sections
store configuration values in static
properties. To help setting and restoring these values, the
classes are built around the concept of \emph{contexts}.
A context is an instance of the \texttt{IContext} interface,
which stores internal data necessary to restore configuration
values to the specific class. Contexts are created using
the static method \texttt{OpenContext} in each of the
specified classes, and destroyed using the \texttt{CloseContext}
method of the \texttt{IContext} instance. When a context
is closed, it restores all configuration
properties to the values they had before the context creation.

As an example, take a look at the following line of code.

\begin{verbatim}
Serializers.PathPrefix = "FolderA";

var context = Serializers.OpenContext();

// All changes after this line will be undo
Serializers.PathPrefix = "FolderB"

// Do your stuff

context.CloseContext();

Debug.Assert(Serializers.PathPrefix == "FolderA");
\end{verbatim}

As you can see if you run this code, the previous value
for the \texttt{PathPrefix} property is restored after the
call to \texttt{CloseContext}. This functionality
becomes increasingly useful if you nest context creations.
The basic idea is that if you need to change some
configuration data inside your algorithm, you do it inside
a context, so that the configuration setted by the caller
is restored before your algorithm ends. 

The \texttt{IContext} interface implements \texttt{IDisposable},
which allows the \texttt{using} idiom to be employed, for
easier manipulation of contexts. The previous example can be
rewritten in the following form;

\begin{verbatim}
Serializers.PathPrefix = "FolderA";

using(Serializers.OpenContext()}
{
    // All changes after this line will be undo
    Serializers.PathPrefix = "FolderB"

    // Do your stuff
}

Debug.Assert(Serializers.PathPrefix == "FolderA");
\end{verbatim}

Notice that the \texttt{using} instruction takes care
of calling \texttt{Dispose}, which in return calls \texttt{CloseContext}.
Since the \texttt{context} variable is unused inside the \texttt{using}
block, there is no need to declare it. This idiom also makes explicit
the contexts nesting which avoids the common mistake of closing contexts
in the wrong order. It is also cleaner since you don't get to see
any alien \texttt{IContext} instance in your code.

%========================================================================
\section{Extension Methods}
%========================================================================

%========================================================================
\section{The \texttt{Scanner} Class}
%========================================================================